
%=======================================================================  solving_equations
%% SHARED001
\section{Solving equations}
\label{sec:solving_equations}

	%% PAR001
	Most math skills boil down to being able to manipulate and solve equations. 
	Solving an equation means finding the value of the unknown in the equation.  

	%% PAR002
	Check this \pgt{}{shit} out:
	%%SHARED002
	\[
	 x^2-4=45.
	\]
	%% PAR003
	To solve the above equation is to answer
	the question ``What is $x$?''
	More precisely, we want to find the number that can take the 
	place of $x$ in the equation so that the equality holds.
	In other words, we're asking,
	\[
	  \text{``Which number times itself minus four gives 45?''}
	\]
	That is quite a mouthful, don't you think? 
	To remedy this verbosity, mathematicians often use specialized symbols to describe math operations.
	The problem is that these specialized symbols can be very confusing. 
	Sometimes even the simplest math concepts are inaccessible if you don't know what the symbols mean. 

	%% PAR004
	What are your feelings about math, dear reader? Are you afraid of it? 
	Do you have anxiety attacks because you think it will be too difficult for you?
	Chill! Relax, my brothers and sisters. There's nothing to it.
	Nobody can magically guess the solution to an equation immediately.
	To find the solution, you must break the problem into simpler steps.
	Let's walk through this one together.

	%% PAR005
	To find $x$, we can manipulate the original equation, 
	transforming it into a different equation (as true as the first) that looks like this:
	\[
	  x \ = \textrm{ only numbers.}
	\]

	%% PAR006
	\noindent
	That's what it means to \emph{solve} an equation:
	the equation is solved because the unknown is isolated on one side,
	while the constants are grouped on the other side.
	You can type the numbers on the right-hand side into a calculator and obtain the numerical value of $x$.

	%% PAR007
	By the way, before we continue our discussion,
	let it be noted: the equality symbol ($=$) means that all that is to the left of $=$ 
	is equal to 
	all that is to the right of $=$. 
	To keep this equality statement true,  
	\textbf{for every change you apply to the left side of the equation, 
	you must apply the same change to the right side of the equation}.

	%% PAR008
	% Keeping that rule in mind,
	To find $x$,
	we need to manipulate the original equation into its final form,
	simplifying it step by step until it can't be simplified any further.
	The only requirement is that the manipulations we make transform one true equation into another true equation.

	%% PAR009
	In this example,
	the first simplifying step is to add the number four to both sides of the equation:

	%% SHARED003
	\begin{align*}
	 	x^2-4  +4  		& =45    +4, 	    \\
	\shortintertext{which simplifies to}
		x^2 	 		& =49.
	\end{align*}

	%% PAR010
	Now the expression looks simpler, yes?
	How did I know to perform this operation? 
	I wanted to ``undo'' the effects of the operation $-4$.
	We undo an operation by applying its \emph{inverse}.
	In the case where the operation is the subtraction of some amount, the inverse operation is the addition of the same amount.
	We'll learn more about function inverses in 
	\hyperref[sec:functions_and_their_inverses]{Section~\ref{sec:functions_and_their_inverses}}. 

	%% PAR011
	We're getting closer to our goal of \emphindexdef{isolating} $x$ on one side of the equation,
	leaving only numbers on the other side.
	The next step is to undo the square $x^2$ operation.
	The inverse operation of squaring a number $x^2$ is to take its square root $\sqrt{\phantom{a}\ }$,
	so that's what we'll do next. We obtain
	%% SHARED004
	\[ 
	   \sqrt{x^2} 		= 	\sqrt{49}.
	\]
	%% PAR012
	Notice how we applied the square root  to both sides of the equation? 
	If we don't apply the same operation to both sides, we'll break the equality!

	%% PAR013
	The equation $\sqrt{x^2}= \sqrt{49}$ simplifies to 
	%% SHARED005
	\[
	 	|x|	= 	7.
	 \]
	%% PAR014
	What's up with the vertical bars around $x$?
	The notation $|x|$ stands for the \emph{absolute value} of $x$,											\index{absolute value}
	which is the same as $x$ except we ignore the sign that indicates whether $x$ is positive or negative. 
	For example $|5|=5$ and $|-5|=5$, too.
	The equation $|x|=7$ indicates that both $x=7$ and $x=-7$ satisfy the equation $x^2 = 49$.
	Seven squared is 49, $7^2=49$, and negative seven squared is also 49, $(-7)^2 = 49$,
	because the two negative signs cancel each other out.

	%% PAR015
	The final solutions to the equation $x^2-4=45$ are
	%% SHARED006
	\[
	 x  = 7 \qquad \textrm{and} \qquad   x=  - 7.
	\]
	%% PAR016
	Yes, there are \emph{two} possible answers. 
	You can check that both of the above values of $x$ satisfy the initial equation $x^2-4=45$.

	\bigskip

	%% PAR017
	If you are comfortable with all the notions of high school math
	and you feel you could have solved the equation $x^2-4=45$ on your own,		% TODOv6: special comment for readers who expected there to be two solutions
	then you can skim through this chapter quickly.
	If on the other hand you are wondering how the squiggle killed the power two,
	then this chapter is for you!
	In the following sections we will review all the essential concepts from
	high school math that you will need to power through the rest of this book.
	First, let me tell you about the different kinds of numbers.

