
\vspace{0.2cm}


\vspace{0.3cm}

\begin{paracol}{2}

	\begin{leftcolumn*}

		\section{Résoudre les equations}
		\label{sec:solving_equations}

		\begin{otherlanguage}{french}
	La plupart des techniques mathématiques peuvent être ramenées à la manipulation et à la résolution d'équations.
	Résoudre une équation, c'est trouver la valeur de l'inconnue dans l'équation.
	% COMMENT: Pour mémoire, une remarque déjà faite: dans cette introduction on annonce, au moins implicitement avec le singulier,
	% qu'il va y avoir UNE solution et crac! dès le premier exemple il y en a deux.
	% Il vaudrait peut-être mieux admettre d'entrée de jeu la possibilité de plusieurs solutions.
	% 		@Gérard: Je pense que c'est correct comme ça---ce serait trop compliqué à dire "le ou les nombres" tout de suite.
	%				Gardez en tête que les gens on la phobie des math donc en ne veux que le début soit aussi simple que possible.


		\end{otherlanguage}

	\end{leftcolumn*}



	\begin{rightcolumn}

		\section{Solving equations}
		\label{sec:solving_equations}


	Most math skills boil down to being able to manipulate and solve equations. 
	Solving an equation means finding the value of the unknown in the equation.  



	\end{rightcolumn}

\end{paracol}

\begin{paracol}{2}

	\begin{leftcolumn*}

		\begin{otherlanguage}{french}
	Regardez ceci:
	\[
	 	x^2-4=45.
	\]
		\end{otherlanguage}

	\end{leftcolumn*}



	\begin{rightcolumn}

	Check this \pgt{}{shit} out:

	\[
	 	x^2-4=45.
	\]

	\end{rightcolumn}

\end{paracol}


\begin{paracol}{2}

	\begin{leftcolumn*}

		\begin{otherlanguage}{french}
	Résoudre cette équation, c'est répondre à la question ‹‹Quelle est la valeur de $x$?››
	Dit de façon plus précise,
	nous voulons trouver le nombre que l'on peut mettre à la place de $x$ dans l'équation pour que l'égalité soit vraie. 
	En d'autres termes, on demande :
	\[
	 	\text{‹‹Quel nombre multiplié par lui-même moins 4 donne 45?››}
	\]
	Ça fait un peu lourd, non? 
	Pour éviter ce verbiage,
	les mathématiciens emploient souvent des symboles spéciaux pour représenter les opérations mathématiques.
	Le problème est que ces symboles peuvent être déroutants.
	Quelquefois, même les concepts mathématiques les plus simples sont inaccessibles si on ne connaît pas la signification des symboles.
	

		\end{otherlanguage}

	\end{leftcolumn*}



	\begin{rightcolumn}

	To solve the above equation is to answer
	the question ``What is $x$?''
	More precisely, we want to find the number that can take the 
	place of $x$ in the equation so that the equality holds.
	In other words, we're asking,
	\[
	  \text{``Which number times itself minus four gives 45?''}
	\]
	That is quite a mouthful, don't you think? 
	To remedy this verbosity, mathematicians often use specialized symbols to describe math operations.
	The problem is that these specialized symbols can be very confusing. 
	Sometimes even the simplest math concepts are inaccessible if you don't know what the symbols mean. 



	\end{rightcolumn}

\end{paracol}

\begin{paracol}{2}

	\begin{leftcolumn*}

		\begin{otherlanguage}{french}
	Chers lecteurs, chères lectrices, quels sont vos sentiments à l'égard des maths?
	Est-ce qu'elles vous effraient?
	Êtes-vous paniqué parce que vous pensez que ça va être trop difficile pour vous?
	Détendez-vous, mes frères et sœurs. Votre réaction est normale.
	Personne ne peut, par miracle, deviner immédiatement la solution d'une équation.
	Pour trouver la solution, vous devez décomposer le problème et procéder étape par étape.
	Faisons ce chemin ensemble.


		\end{otherlanguage}

	\end{leftcolumn*}



	\begin{rightcolumn}

	What are your feelings about math, dear reader? Are you afraid of it? 
	Do you have anxiety attacks because you think it will be too difficult for you?
	Chill! Relax, my brothers and sisters. There's nothing to it.
	Nobody can magically guess the solution to an equation immediately.
	To find the solution, you must break the problem into simpler steps.
	Let's walk through this one together.



	\end{rightcolumn}

\end{paracol}

\begin{paracol}{2}

	\begin{leftcolumn*}

		\begin{otherlanguage}{french}
	Pour trouver $x$,
	nous pouvons manipuler l'équation initiale pour la transformer en une équation différente qui soit aussi vraie que la première,
	mais se présente sous la forme: 
	\[ 
		x \ = \textrm{ rien que des nombres.}
	\]

		\end{otherlanguage}

	\end{leftcolumn*}



	\begin{rightcolumn}

	To find $x$, we can manipulate the original equation, 
	transforming it into a different equation (as true as the first) that looks like this:
	\[
	  x \ = \textrm{ only numbers.}
	\]



	\end{rightcolumn}

\end{paracol}

\begin{paracol}{2}

	\begin{leftcolumn*}

		\begin{otherlanguage}{french}
	C'est ça qu'on appelle \emph{résoudre} une équation.
	L'équation est résolue parce que l'inconnue est isolée d'un seul côté,
	tandis que les nombres sont regroupés de l'autre côté.
	Vous pouvez entrer dans votre calculateur les nombres qui apparaissent du côté droit de l'équation et 
	obtenir la valeur numérique de $x$.


		\end{otherlanguage}

	\end{leftcolumn*}



	\begin{rightcolumn}

	\noindent
	That's what it means to \emph{solve} an equation:
	the equation is solved because the unknown is isolated on one side,
	while the constants are grouped on the other side.
	You can type the numbers on the right-hand side into a calculator and obtain the numerical value of $x$.



	\end{rightcolumn}

\end{paracol}

\begin{paracol}{2}

	\begin{leftcolumn*}

		\begin{otherlanguage}{french}
	Avant de continuer notre discussion, remarquons ceci:
	le signe d'égalité ($=$) signifie que tout ce qui est à gauche du signe $=$ est égal à tout ce qui est à droite.
        Pour que cette qualité  du signe $=$ reste vraie,
        \textbf{pour tout changement que l'on fait du côté gauche de l'équation on doit faire le même changement du côté droit}. 
	Gardons présente à l'esprit la règle suivante:
	pour trouver $x$ on doit manipuler l'équation originale jusqu'à obtenir la forme finale,
	en simplifiant pas à pas jusqu'à ce qu'on ne puisse plus simplifier davantage.
	Les manipulations que nous faisons transforment une équation vraie en une autre équation vraie. %qui a la même signification.


		\end{otherlanguage}

	\end{leftcolumn*}



	\begin{rightcolumn}

	By the way, before we continue our discussion,
	let it be noted: the equality symbol ($=$) means that all that is to the left of $=$ 
	is equal to 
	all that is to the right of $=$. 
	To keep this equality statement true,  
	\textbf{for every change you apply to the left side of the equation, 
	you must apply the same change to the right side of the equation}.



	\end{rightcolumn}

\end{paracol}

\begin{paracol}{2}

	\begin{leftcolumn*}

		\begin{otherlanguage}{french}
	Dans notre exemple,
	la première étape de simplification sera d'ajouter le nombre 4 aux deux côtés de l'équation:

	\[
	 	x^2-4  +4  		=	45    +4,
	\]
	qui se simplifie en
	\[
		x^2 	 		=	49.
	\]


		\end{otherlanguage}

	\end{leftcolumn*}



	\begin{rightcolumn}

	In this example,
	the first simplifying step is to add the number four to both sides of the equation:


	\[
	 	x^2-4  +4  		=	45    +4,
	\]
	which simplifies to
	\[
		x^2 	 		=	49.
	\]



	\end{rightcolumn}

\end{paracol}


\begin{paracol}{2}

	\begin{leftcolumn*}

		\begin{otherlanguage}{french}
	\noindent
	Maintenant l'expression paraît plus simple, non?
	Pourquoi ai-je fait cette opération? 
	Je voulais ‹‹annuler›› les effets de l'opération $-4$.
	Nous  annulons une opération en faisant l'opération \emph{inverse}.
	Lorsque l'opération est la soustraction d'une certaine quantité, l'opération inverse est l'addition de la même quantité.
	Nous en apprendrons plus au sujet des fonctions inverses dans la section~\ref{sec:functions_and_their_inverses}.


		\end{otherlanguage}

	\end{leftcolumn*}



	\begin{rightcolumn}

	Now the expression looks simpler, yes?
	How did I know to perform this operation? 
	I wanted to ``undo'' the effects of the operation $-4$.
	We undo an operation by applying its \emph{inverse}.
	In the case where the operation is the subtraction of some amount, the inverse operation is the addition of the same amount.
	We'll learn more about function inverses in 
	\hyperref[sec:functions_and_their_inverses]{Section~\ref{sec:functions_and_their_inverses}}. 



	\end{rightcolumn}

\end{paracol}

\begin{paracol}{2}

	\begin{leftcolumn*}

		\begin{otherlanguage}{french}
	Nous approchons de notre but: \emphindexdef{isoler} $x$ d'un côté de l'équation,
	ne laissant que des nombres de l'autre côté.
	Dans l'étape suivante on annule l'opération élever au carré $x^2$.
	L'opération inverse d'élever un nombre au carré ($x^2$) est de prendre sa racine carrée $\sqrt{\phantom{a}\ }$.
	C'est donc ce que nous faisons maintenant et nous obtenons:

	\[ 
	   \sqrt{x^2} 		= 	\sqrt{49}.
	\]
		\end{otherlanguage}

	\end{leftcolumn*}



	\begin{rightcolumn}

	We're getting closer to our goal of \emphindexdef{isolating} $x$ on one side of the equation,
	leaving only numbers on the other side.
	The next step is to undo the square $x^2$ operation.
	The inverse operation of squaring a number $x^2$ is to take its square root $\sqrt{\phantom{a}\ }$,
	so that's what we'll do next. We obtain

	\[ 
	   \sqrt{x^2} 		= 	\sqrt{49}.
	\]

	\end{rightcolumn}

\end{paracol}


\begin{paracol}{2}

	\begin{leftcolumn*}

		\begin{otherlanguage}{french}
	Observez que nous avons pris la racine carrée des deux côtés de l'équation.
	Si nous n'avions pas fait la même opération des deux côtés, nous n'aurions plus l'égalité!


		\end{otherlanguage}

	\end{leftcolumn*}



	\begin{rightcolumn}

	Notice how we applied the square root  to both sides of the equation? 
	If we don't apply the same operation to both sides, we'll break the equality!



	\end{rightcolumn}

\end{paracol}

\begin{paracol}{2}

	\begin{leftcolumn*}

		\begin{otherlanguage}{french}
	L'équation $\sqrt{x^2}= \sqrt{49}$ se simplifie en: 

	\[
	 	|x|	= 	7.
	\]
		\end{otherlanguage}

	\end{leftcolumn*}



	\begin{rightcolumn}

	The equation $\sqrt{x^2}= \sqrt{49}$ simplifies to 

	\[
	 	|x|	= 	7.
	\]

	\end{rightcolumn}

\end{paracol}


\begin{paracol}{2}

	\begin{leftcolumn*}

		\begin{otherlanguage}{french}
	Que représentent ces barres verticales autour de $x$?
	La notation $|x|$ signifie \emph{valeur absolue} de $x$,											\index{valeur absolue}
	qui est ce que l'on obtient à partir de $x$ lorsqu'on ignore le signe qui indique si $x$ est positif ou négatif. 
	Par exemple $|5|=5$ mais $|-5|=5$ aussi.
	L'équation $|x|=7$ veut dire que $x=7$ et $x=-7$ satisfont tous deux l'équation $x^2 = 49$.
	Sept au carré est $49$, $7^2=49$, et $-7$ au carré est aussi $49$, 
	$(-7)^2 = 49$,
	parce que les deux signes moins s'annulent l'un, l'autre dans la multiplication $(-7)(-7)=49$.


		\end{otherlanguage}

	\end{leftcolumn*}



	\begin{rightcolumn}

	What's up with the vertical bars around $x$?
	The notation $|x|$ stands for the \emph{absolute value} of $x$,											\index{absolute value}
	which is the same as $x$ except we ignore the sign that indicates whether $x$ is positive or negative. 
	For example $|5|=5$ and $|-5|=5$, too.
	The equation $|x|=7$ indicates that both $x=7$ and $x=-7$ satisfy the equation $x^2 = 49$.
	Seven squared is 49, $7^2=49$, and negative seven squared is also 49, $(-7)^2 = 49$,
	because the two negative signs cancel each other out.



	\end{rightcolumn}

\end{paracol}

\begin{paracol}{2}

	\begin{leftcolumn*}

		\begin{otherlanguage}{french}
	Les solutions de l'équation $x^2-4=45$ sont donc
	\[
	 	x  = 7 \qquad \textrm{et} \qquad   x=  - 7.
	\]
		\end{otherlanguage}

	\end{leftcolumn*}



	\begin{rightcolumn}

	The final solutions to the equation $x^2-4=45$ are
	\[
	 	x  = 7 \qquad \textrm{and} \qquad   x=  - 7.
	\]

	\end{rightcolumn}

\end{paracol}


\begin{paracol}{2}

	\begin{leftcolumn*}

		\begin{otherlanguage}{french}
	Eh oui, il y a \emph{deux} réponses possibles!
	Vous pouvez vérifier en faisant le calcul que les deux valeurs de $x$ ci-dessus satisfont l'équation initiale $x^2-4=45$.

	\bigskip


		\end{otherlanguage}

	\end{leftcolumn*}



	\begin{rightcolumn}

	Yes, there are \emph{two} possible answers. 
	You can check that both of the above values of $x$ satisfy the initial equation $x^2-4=45$.

	\bigskip



	\end{rightcolumn}

\end{paracol}

\begin{paracol}{2}

	\begin{leftcolumn*}

		\begin{otherlanguage}{french}
	Si vous vous sentez à l'aise avec toutes les notions de maths utilisées dans cette section
	et si vous pensez que vous auriez pu résoudre l'équation $x^2-4=45$ par vous-même,
	alors vous pouvez vous contenter de regarder rapidement ce chapitre.
	Si, au contraire, vous ne vous sentez pas bien à l'aise dans les étapes de la résolution ci-dessus alors ce chapitre est pour vous!
	Dans les sections suivantes nous allons réviser les concepts essentiels des maths dont vous aurez besoin pour maîtriser le reste de ce livre.
	
	Pour commencer laissez moi vous parler des différents types de nombres.


		\end{otherlanguage}

	\end{leftcolumn*}



	\begin{rightcolumn}

	If you are comfortable with all the notions of high school math
	and you feel you could have solved the equation $x^2-4=45$ on your own,		% TODOv6: special comment for readers who expected there to be two solutions
	then you can skim through this chapter quickly.
	If on the other hand you are wondering how the squiggle killed the power two,
	then this chapter is for you!
	In the following sections we will review all the essential concepts from
	high school math that you will need to power through the rest of this book.
	First, let me tell you about the different kinds of numbers.



	\end{rightcolumn}

\end{paracol}
