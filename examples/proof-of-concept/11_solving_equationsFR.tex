
%=======================================================================  solving_equations
%% SHARED001
\section{Résoudre les equations}
\label{sec:solving_equations}

	%% PAR001
	La plupart des techniques mathématiques peuvent être ramenées à la manipulation et à la résolution d'équations.
	Résoudre une équation, c'est trouver la valeur de l'inconnue dans l'équation.
	% COMMENT: Pour mémoire, une remarque déjà faite: dans cette introduction on annonce, au moins implicitement avec le singulier,
	% qu'il va y avoir UNE solution et crac! dès le premier exemple il y en a deux.
	% Il vaudrait peut-être mieux admettre d'entrée de jeu la possibilité de plusieurs solutions.
	% 		@Gérard: Je pense que c'est correct comme ça---ce serait trop compliqué à dire "le ou les nombres" tout de suite.
	%				Gardez en tête que les gens on la phobie des math donc en ne veux que le début soit aussi simple que possible.

	%% PAR002
	Regardez ceci:
	%%SHARED002
	\[
	 	x^2-4=45.
	\]
	%% PAR003
	Résoudre cette équation, c'est répondre à la question ‹‹Quelle est la valeur de $x$?››
	Dit de façon plus précise,
	nous voulons trouver le nombre que l'on peut mettre à la place de $x$ dans l'équation pour que l'égalité soit vraie. 
	En d'autres termes, on demande :
	\[
	 	\text{‹‹Quel nombre multiplié par lui-même moins 4 donne 45?››}
	\]
	Ça fait un peu lourd, non? 
	Pour éviter ce verbiage,
	les mathématiciens emploient souvent des symboles spéciaux pour représenter les opérations mathématiques.
	Le problème est que ces symboles peuvent être déroutants.
	Quelquefois, même les concepts mathématiques les plus simples sont inaccessibles si on ne connaît pas la signification des symboles.
	
	%% PAR004
	Chers lecteurs, chères lectrices, quels sont vos sentiments à l'égard des maths?
	Est-ce qu'elles vous effraient?
	Êtes-vous paniqué parce que vous pensez que ça va être trop difficile pour vous?
	Détendez-vous, mes frères et sœurs. Votre réaction est normale.
	Personne ne peut, par miracle, deviner immédiatement la solution d'une équation.
	Pour trouver la solution, vous devez décomposer le problème et procéder étape par étape.
	Faisons ce chemin ensemble.

	%% PAR005
	Pour trouver $x$,
	nous pouvons manipuler l'équation initiale pour la transformer en une équation différente qui soit aussi vraie que la première,
	mais se présente sous la forme: 
	\[ 
		x \ = \textrm{ rien que des nombres.}
	\]
	%% PAR006
	C'est ça qu'on appelle \emph{résoudre} une équation.
	L'équation est résolue parce que l'inconnue est isolée d'un seul côté,
	tandis que les nombres sont regroupés de l'autre côté.
	Vous pouvez entrer dans votre calculateur les nombres qui apparaissent du côté droit de l'équation et 
	obtenir la valeur numérique de $x$.

	%% PAR007
	Avant de continuer notre discussion, remarquons ceci:
	le signe d'égalité ($=$) signifie que tout ce qui est à gauche du signe $=$ est égal à tout ce qui est à droite.
        Pour que cette qualité  du signe $=$ reste vraie,
        \textbf{pour tout changement que l'on fait du côté gauche de l'équation on doit faire le même changement du côté droit}. 
	Gardons présente à l'esprit la règle suivante:
	pour trouver $x$ on doit manipuler l'équation originale jusqu'à obtenir la forme finale,
	en simplifiant pas à pas jusqu'à ce qu'on ne puisse plus simplifier davantage.
	Les manipulations que nous faisons transforment une équation vraie en une autre équation vraie. %qui a la même signification.

	%% PAR009
	Dans notre exemple,
	la première étape de simplification sera d'ajouter le nombre 4 aux deux côtés de l'équation:
	%% SHARED003
	\[
	 	x^2-4  +4  		=	45    +4,
	\]
	qui se simplifie en
	\[
		x^2 	 		=	49.
	\]

	%% PAR010
	\noindent
	Maintenant l'expression paraît plus simple, non?
	Pourquoi ai-je fait cette opération? 
	Je voulais ‹‹annuler›› les effets de l'opération $-4$.
	Nous  annulons une opération en faisant l'opération \emph{inverse}.
	Lorsque l'opération est la soustraction d'une certaine quantité, l'opération inverse est l'addition de la même quantité.
	Nous en apprendrons plus au sujet des fonctions inverses dans la section~\ref{sec:functions_and_their_inverses}.

	%% PAR011
	Nous approchons de notre but: \emphindexdef{isoler} $x$ d'un côté de l'équation,
	ne laissant que des nombres de l'autre côté.
	Dans l'étape suivante on annule l'opération élever au carré $x^2$.
	L'opération inverse d'élever un nombre au carré ($x^2$) est de prendre sa racine carrée $\sqrt{\phantom{a}\ }$.
	C'est donc ce que nous faisons maintenant et nous obtenons:
	%% SHARED004
	\[ 
	   \sqrt{x^2} 		= 	\sqrt{49}.
	\]
	%% PAR012
	Observez que nous avons pris la racine carrée des deux côtés de l'équation.
	Si nous n'avions pas fait la même opération des deux côtés, nous n'aurions plus l'égalité!

	%% PAR013
	L'équation $\sqrt{x^2}= \sqrt{49}$ se simplifie en: 
	%% SHARED005
	\[
	 	|x|	= 	7.
	\]
	%% PAR014
	Que représentent ces barres verticales autour de $x$?
	La notation $|x|$ signifie \emph{valeur absolue} de $x$,											\index{valeur absolue}
	qui est ce que l'on obtient à partir de $x$ lorsqu'on ignore le signe qui indique si $x$ est positif ou négatif. 
	Par exemple $|5|=5$ mais $|-5|=5$ aussi.
	L'équation $|x|=7$ veut dire que $x=7$ et $x=-7$ satisfont tous deux l'équation $x^2 = 49$.
	Sept au carré est $49$, $7^2=49$, et $-7$ au carré est aussi $49$, 
	$(-7)^2 = 49$,
	parce que les deux signes moins s'annulent l'un, l'autre dans la multiplication $(-7)(-7)=49$.

	%% PAR015
	Les solutions de l'équation $x^2-4=45$ sont donc
	%% SHARED006
	\[
	 	x  = 7 \qquad \textrm{et} \qquad   x=  - 7.
	\]
	%% PAR016
	Eh oui, il y a \emph{deux} réponses possibles!
	Vous pouvez vérifier en faisant le calcul que les deux valeurs de $x$ ci-dessus satisfont l'équation initiale $x^2-4=45$.

	\bigskip

	%% PAR017
	Si vous vous sentez à l'aise avec toutes les notions de maths utilisées dans cette section
	et si vous pensez que vous auriez pu résoudre l'équation $x^2-4=45$ par vous-même,
	alors vous pouvez vous contenter de regarder rapidement ce chapitre.
	Si, au contraire, vous ne vous sentez pas bien à l'aise dans les étapes de la résolution ci-dessus alors ce chapitre est pour vous!
	Dans les sections suivantes nous allons réviser les concepts essentiels des maths dont vous aurez besoin pour maîtriser le reste de ce livre.
	
	Pour commencer laissez moi vous parler des différents types de nombres.

